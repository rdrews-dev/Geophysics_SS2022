\documentclass[a4paper,12pt,fleqn]{article}
%All Layout Packages are defined in the Header.tex
\usepackage[utf8]{inputenc}
\usepackage[english]{babel}
\usepackage{fancyhdr}

%This is to print solutions or not
\usepackage{ifthen}

%This is to include src code
\usepackage{listings}
\lstset{ %
  backgroundcolor=\color{white},   % choose the background color; you must add \usepackage{color} or \usepackage{xcolor}
  basicstyle=\footnotesize,        % the size of the fonts that are used for the code
  breakatwhitespace=false,         % sets if automatic breaks should only happen at whitespace
  breaklines=true,                 % sets automatic line breaking
  captionpos=b,                    % sets the caption-position to bottom
  commentstyle=\color{commentsColor}\textit,    % comment style
  deletekeywords={...},            % if you want to delete keywords from the given language
  escapeinside={\%*}{*)},          % if you want to add LaTeX within your code
  extendedchars=true,              % lets you use non-ASCII characters; for 8-bits encodings only, does not work with UTF-8
  frame=tb,	                   	   % adds a frame around the code
  keepspaces=true,                 % keeps spaces in text, useful for keeping indentation of code (possibly needs columns=flexible)
  keywordstyle=\color{keywordsColor}\bfseries,       % keyword style
  language=Python,                 % the language of the code (can be overrided per snippet)
  otherkeywords={*,...},           % if you want to add more keywords to the set
  numbers=left,                    % where to put the line-numbers; possible values are (none, left, right)
  numbersep=5pt,                   % how far the line-numbers are from the code
  numberstyle=\tiny\color{commentsColor}, % the style that is used for the line-numbers
  rulecolor=\color{black},         % if not set, the frame-color may be changed on line-breaks within not-black text (e.g. comments (green here))
  showspaces=false,                % show spaces everywhere adding particular underscores; it overrides 'showstringspaces'
  showstringspaces=false,          % underline spaces within strings only
  showtabs=false,                  % show tabs within strings adding particular underscores
  stepnumber=1,                    % the step between two line-numbers. If it's 1, each line will be numbered
  stringstyle=\color{stringColor}, % string literal style
  tabsize=2,	                   % sets default tabsize to 2 spaces
  title=\lstname,                  % show the filename of files included with \lstinputlisting; also try caption instead of title
  columns=fixed                    % Using fixed column width (for e.g. nice alignment)
}


%Use this to customize margins
\usepackage[
  top=2cm,
  bottom=2cm,
  left=1cm,
  right=1cm,
  headheight=17pt, % as per the warning by fancyhdr
  includehead,includefoot,
  heightrounded, % to avoid spurious underfull messages
]{geometry} 

%Use this to customize headers
\pagestyle{fancy}
\fancyhf{}
\fancyhead[LE,RO]{Version: \today}
\fancyhead[RE,LO]{Geophysics Exercises}
\fancyfoot[CE,CO]{\leftmark}
\fancyfoot[LE,RO]{\thepage}
\usepackage{graphicx}
\renewcommand{\headrulewidth}{2pt}
\renewcommand{\footrulewidth}{1pt}


%Use this to customize tables
\usepackage[table]{xcolor}
\usepackage{tabularx}
\usepackage{booktabs}


\setlength{\arrayrulewidth}{1mm}
\setlength{\tabcolsep}{18pt}
\renewcommand{\arraystretch}{1.5}


%Use this to customize colors
\definecolor{myblue}{rgb}{0.0, 0.53, 0.74}
\definecolor{myred}{rgb}{1.0, 0.8, 0.89}
\definecolor{babyblueeyes}{rgb}{0.63, 0.79, 0.95}
\definecolor{beaublue}{rgb}{0.74, 0.83, 0.9}
\definecolor{bluegray}{rgb}{0.4, 0.6, 0.8}
\definecolor{commentsColor}{rgb}{0.497495, 0.497587, 0.497464}
\definecolor{keywordsColor}{rgb}{0.000000, 0.000000, 0.635294}
\definecolor{stringColor}{rgb}{0.558215, 0.000000, 0.135316}
\definecolor{burgundy}{rgb}{0.5, 0.0, 0.13}
%Here some pre-defined commands
\newcommand{\ProtocolTable}[6]
{
\begin{table}[]
\begin{tabular}{|p{1.8cm} p{3.8cm} p{1.8cm} p{2.0cm}|}
\hline
 \rowcolor{beaublue}\textbf{Name}:&\multicolumn{3}{l}{#1} \\
 \rowcolor{beaublue}\textbf{Folder:}&\multicolumn{3}{l}{#2} \\
 \rowcolor{beaublue}\textbf{Instrument:}&#3&\textbf{Date:}&#4 \\
  \rowcolor{beaublue}\textbf{Operator:}&#5&\textbf{Location:}&#6\\
\hline
\end{tabular}
\end{table}
}
\usepackage{enumitem}
\usepackage{blindtext}
\usepackage{afterpage}
\usepackage{lipsum,lmodern}
\usepackage[most]{tcolorbox}
\tcbuselibrary{skins}
\usepackage{hyperref}

%no indents at paragraphs 
\setlength\parindent{0pt}

%% Oh yes. TIKZ pictures / graphs
%% -------------------------------------
\usepackage{tikz}
\usepackage{pgfplots}
\usepackage{tikz-3dplot}
\usetikzlibrary{calc,fadings,decorations.pathreplacing,shadings,intersections}


\usepackage{amsmath}
\usepackage{booktabs,array,tabularx}
\usepackage{esvect}

\begin{document}



 \section{Maxwell Equation and Electromagnetic Induction}
 \textbf{Version:} \today 

 \textbf{Motivation:} Ten most important equations that changed the world. Among those:

\begin{eqnarray*}
  &a^2+b^2 = c^2 & \text{(Pythagoras)}\\
  &e^{i\pi}+1=0 \quad &\text{(Euler)}\\
  &\vec{F} = m\vec{a}; \vec{F} = G\frac{mM}{r^2} & \text{(Newton)} \\
  &(... \text{Einstein, Schrödinger})
 \end{eqnarray*}
 
 and Maxwell:

 \begin{eqnarray*}
  \nabla \cdot \vec{\mathbf{D}}  &=& \rho \qquad  \text{(Gauss)}\\
  \nabla \cdot \vec{\mathbf{B}}  &=& 0 \qquad  \text{(Gauss)} \\
  \nabla \times \vec{\mathbf{E}} &=& -\frac{\partial \vec{\mathbf{B}}} {\partial t} \qquad \text{(Faraday)}\\
  \nabla \times \vec{\mathbf{H}} &=& \vec{\mathbf{J}} + \frac{\partial \vec{\mathbf{D}}} {\partial t} \qquad \text{(Amp\'{e}re-Maxwell)}\\
  \vec{\mathbf{D}} &=& \varepsilon \varepsilon_0\vec{\mathbf{E}} \qquad  \text{(materials: electric field, dielectric field)} \\
  \vec{\mathbf{H}} &=& \mu \mu_0\vec{\mathbf{B}}  \qquad \text{(materials: magnetizing field, magnetic induction)} \\
  \vec{j} &=& \sigma \vec{\mathbf{E}}  \qquad \text{(Ohm's law)} \\
 \end{eqnarray*}

 and Integral:

 \begin{eqnarray*}
  \int \int_{\partial \Omega} \vec{\mathbf{D}}\cdot d\vec{\mathbf{S}} &=& \int\int\int_{\Omega}\rho \qquad  \text{(Gauss)}\\
  \int \int_{\partial \Omega} \vec{\mathbf{B}}\cdot d\vec{\mathbf{S}} &=& 0 \qquad  \text{(Gauss)} \\
  \int_{\partial \Sigma} \vec{\mathbf{E}}\cdot d\vec{\mathbf{l}} &=& -\frac{\partial} {\partial t} \int\int_{\Sigma}\vec{\mathbf{B}}\cdot{d\vec{S}}\qquad \text{(Faraday)}\\
  \int_{\partial \Sigma} \vec{\mathbf{H}} \cdot d\vec{\mathbf{l}} &=& \int\int_{\Sigma} \vec{\mathbf{J}}\cdot d\vec{\mathbf{S}} + \frac{\partial} {\partial t} \int\int_{\Sigma}\vec{\mathbf{D}}\cdot d\vec{\mathbf{S}} \qquad \text{(Amp\'{e}re-Maxwell)}\\
  \Omega: \text{Volume} \\
  \Sigma: \text{Surface} \\
  \partial \Omega: \text{Surface of volume} \\
  \partial \Sigma: \text{Edge of surface} \\
 \end{eqnarray*}
 
Have a picture in mind for each one of them:
(1) Source of a static E-field.
(2) Dipole b-Field.
(3) Induction with Magnet.
(4) Displacement Currents and Bio-Savart law

Integral vs. Differential form: EMF, loops, LENZ law\\

Principle of self-induction\\

Table resistance frequency dependency \\

$Z_L = j \omega L$  V lags I
$Z_c = \frac{1}{i \omega C}$ I lags V
$Z_R = R$ in phase \\

Classification of electrical methods \\


Principles of the Slingram Method\\

$I_{l1} = I_1 e^{i\omega t}$


$V_{l3} = L_{13}\frac{d I_{l1}}{dt} = i\omega L_{13}I_1e^{i\omega t}$
$V_{l2} = L_{12}\frac{d I_{l1}}{dt} = i\omega L_{12}I_1e^{i\omega t}$

What is $I_2$ with a R-L Subsurface model? Ohms law:

$Z_{l2} = R + i\omega L_{l2}$
$V_{l2} = (R + i\omega L_{l2})I_2$

Induced voltage is balanced by inductance and resistance:

$i \omega L_{12} I_1 e^{i\omega t} + (R + i\omega L_{l2})I_2 = 0$\\
$I_{l2} = I_2e^{i\omega } = \frac{-i\omega L_{12}}{R_{l2}+i\omega L_{l2}}I_1e^{i\omega t}$
$I_{l2} = I_2e^{i\omega t} = \frac{-i\omega \frac{L_{l2}}{R_{l2}}}{1+i\omega \frac{L_{l2}}{R_{l2}}}\frac{L_{12}}{L_{l2}}I_1e^{i\omega t}$\\

Il2 will produce a secondary B. How does that appear in Loop 3:

$V_{l3} = L_{13}\frac{dI_l1}{dt}$ primary
$V_{l3} = L_{23}\frac{dI_l2}{dt}$ secondary

$\frac{U_s}{U_p}= -\frac{L_{12}L_{23}}{L_{13}L_{l2}}\left(\frac{i\omega \frac{L_{l2}}{R_{l2}}}{1+i\omega\frac{L_{l2}}{R_{l2}}}\right)$

induction number:
$
\alpha = \omega \frac{L_{l2}}{R_{l2}}
$
helps to write the complex number in standard form:
$\frac{U_s}{U_p}== -\frac{L_{12}L_{23}}{L_{13}L_{l2}} \left(\frac{1}{1+\alpha^2}(\alpha^2+i\alpha) \right)$




\end{document}




