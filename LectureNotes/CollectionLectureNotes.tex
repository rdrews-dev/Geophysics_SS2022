\documentclass[a4paper,12pt,fleqn]{article}
%All Layout Packages are defined in the Header.tex
\include{Header}
\usepackage{amsmath}
\usepackage{booktabs,array,tabularx}
\usepackage{esvect}

\begin{document}



 \section{Maxwell Equation and Electromagnetic Induction}
 \textbf{Version:} \today 

 \textbf{Motivation:} Ten most important equations that changed the world. Among those:

\begin{eqnarray*}
  &a^2+b^2 = c^2 & \text{(Pythagoras)}\\
  &e^{i\pi}+1=0 \quad &\text{(Euler)}\\
  &\vec{F} = m\vec{a}; \vec{F} = G\frac{mM}{r^2} & \text{(Newton)} \\
  &(... \text{Einstein, Schrödinger})
 \end{eqnarray*}
 
 and Maxwell:

 \begin{eqnarray*}
  \nabla \cdot \vec{\mathbf{D}}  &=& \rho \qquad  \text{(Gauss)}\\
  \nabla \cdot \vec{\mathbf{B}}  &=& 0 \qquad  \text{(Gauss)} \\
  \nabla \times \vec{\mathbf{E}} &=& -\frac{\partial \vec{\mathbf{B}}} {\partial t} \qquad \text{(Faraday)}\\
  \nabla \times \vec{\mathbf{H}} &=& \vec{\mathbf{J}} + \frac{\partial \vec{\mathbf{D}}} {\partial t} \qquad \text{(Amp\'{e}re-Maxwell)}\\
  \vec{\mathbf{D}} &=& \varepsilon \varepsilon_0\vec{\mathbf{E}} \qquad  \text{(materials: electric field, dielectric field)} \\
  \vec{\mathbf{H}} &=& \mu \mu_0\vec{\mathbf{B}}  \qquad \text{(materials: magnetizing field, magnetic induction)} \\
  \vec{j} &=& \sigma \vec{\mathbf{E}}  \qquad \text{(Ohm's law)} \\
 \end{eqnarray*}

 and Integral:

 \begin{eqnarray*}
  \int \int_{\partial \Omega} \vec{\mathbf{D}}\cdot d\vec{\mathbf{S}} &=& \int\int\int_{\Omega}\rho \qquad  \text{(Gauss)}\\
  \int \int_{\partial \Omega} \vec{\mathbf{B}}\cdot d\vec{\mathbf{S}} &=& 0 \qquad  \text{(Gauss)} \\
  \int_{\partial \Sigma} \vec{\mathbf{E}}\cdot d\vec{\mathbf{l}} &=& -\frac{\partial} {\partial t} \int\int_{\Sigma}\vec{\mathbf{B}}\cdot{d\vec{S}}\qquad \text{(Faraday)}\\
  \int_{\partial \Sigma} \vec{\mathbf{H}} \cdot d\vec{\mathbf{l}} &=& \int\int_{\Sigma} \vec{\mathbf{J}}\cdot d\vec{\mathbf{S}} + \frac{\partial} {\partial t} \int\int_{\Sigma}\vec{\mathbf{D}}\cdot d\vec{\mathbf{S}} \qquad \text{(Amp\'{e}re-Maxwell)}\\
  \Omega: \text{Volume} \\
  \Sigma: \text{Surface} \\
  \partial \Omega: \text{Surface of volume} \\
  \partial \Sigma: \text{Edge of surface} \\
 \end{eqnarray*}
 
Have a picture in mind for each one of them:
(1) Source of a static E-field.
(2) Dipole b-Field.
(3) Induction with Magnet.
(4) Displacement Currents and Bio-Savart law

Integral vs. Differential form: EMF, loops, LENZ law\\

Principle of self-induction\\

Table resistance frequency dependency \\

$Z_L = j \omega L$  V lags I
$Z_c = \frac{1}{i \omega C}$ I lags V
$Z_R = R$ in phase \\

Classification of electrical methods \\


Principles of the Slingram Method\\

$I_{l1} = I_1 e^{i\omega t}$


$V_{l3} = L_{13}\frac{d I_{l1}}{dt} = i\omega L_{13}I_1e^{i\omega t}$
$V_{l2} = L_{12}\frac{d I_{l1}}{dt} = i\omega L_{12}I_1e^{i\omega t}$

What is $I_2$ with a R-L Subsurface model? Ohms law:

$Z_{l2} = R + i\omega L_{l2}$
$V_{l2} = (R + i\omega L_{l2})I_2$

Induced voltage is balanced by inductance and resistance:

$i \omega L_{12} I_1 e^{i\omega t} + (R + i\omega L_{l2})I_2 = 0$\\
$I_{l2} = I_2e^{i\omega } = \frac{-i\omega L_{12}}{R_{l2}+i\omega L_{l2}}I_1e^{i\omega t}$
$I_{l2} = I_2e^{i\omega t} = \frac{-i\omega \frac{L_{l2}}{R_{l2}}}{1+i\omega \frac{L_{l2}}{R_{l2}}}\frac{L_{12}}{L_{l2}}I_1e^{i\omega t}$\\

Il2 will produce a secondary B. How does that appear in Loop 3:

$V_{l3} = L_{13}\frac{dI_l1}{dt}$ primary
$V_{l3} = L_{23}\frac{dI_l2}{dt}$ secondary

$\frac{U_s}{U_p}= -\frac{L_{12}L_{23}}{L_{13}L_{l2}}\left(\frac{i\omega \frac{L_{l2}}{R_{l2}}}{1+i\omega\frac{L_{l2}}{R_{l2}}}\right)$

induction number:
$
\alpha = \omega \frac{L_{l2}}{R_{l2}}
$
helps to write the complex number in standard form:
$\frac{U_s}{U_p}== -\frac{L_{12}L_{23}}{L_{13}L_{l2}} \left(\frac{1}{1+\alpha^2}(\alpha^2+i\alpha) \right)$

\section{Seismics}
$$
\frac{\partial^2 \Psi}{\partial t^2 \Psi} = v^2 \frac{\partial^2 \Psi}{\partial x^2 \Psi}
$$
\begin{itemize}
  \item This is an (undamped) wave equation.
  \item This is a second order (hyperbolic) partial differential equation.
  \item It is a function of space $x$ and time $t$.
\end{itemize}
\subsection{General solution}
$$\Psi = f(x+vt) + f(x-vt)$$
Exampel: $\Psi(x,t) = e^{k(x-vt)}, \sin(x-vt), (x-vt)^3$.

However, only one solution complies with the given initial and boundary conditions. What is $v$?
\\
Draw Picture 4.4 Telford of $\Psi$  of wave form $\Psi$ on y-axis with an $x_0$ and $t_0+\Delta t$ marked. 
\\
$$t_0+\Delta t: \Psi_1(x_0+\Delta_x,t+t_0) = f(x_0+\Delta_x-v(t_0+\Delta t))$$
$$t_0: \Psi_1(x_0,t) = f(x_0+\Delta_x-vt_0)$$
$$
x_0 - vt_0 = x_0 + \Delta x -v(t_0+\Delta t)
$$
\begin{itemize}
\item $x \pm vt$ is the phase
\item $x \pm vt = const.$  wavefronts
\item normal to the wavefront is the raypath
\end{itemize}

\subsection{Specific solutions}
\subsubsection{Spherical waves}
$$
\frac{\partial^2 \Psi}{\partial t^2 \Psi} = \frac{v^2}{r^2}\frac{\partial}{\partial r}\left(r^2\frac{\partial\Psi}{\partial r}\right)
$$
Solution:
$$
\psi(x,t) = \frac{1}{r}f(r-vt)
$$
Spherical waves the circles are wavefronts (i.e. lines of constant phase). Raypath are thus along $r$. In the farfield spherical waves can be approximated with plane waves.

\textit{draw Figure 4.6 plane waves and spherical waves}
\subsubsection{Harmonic waves}
\begin{eqnarray}
\psi(x,t) = A\cos(k(x-vt))\\
\psi(x,t) = A/r \cos(k(x-vt))
\end{eqnarray}

At fixed t if x increases by 2$\pi$/k then everything repeats so 
$$
\lambda = \frac{2\pi}{k} \text{wavelength}
$$ 
At fixed x it varies harmonically with time from -A to A (or -A/r to A/r). A is the amplitude.

At fixed x repetition by T if  $kVT=2\pi = 2\pi(vT/\lambda)$

\begin{eqnarray}
  T = \lambda/v \\
  f = 1/T \\
  v = \lambda f
\end{eqnarray}
With T period, f frequency, v phase velocity. Commonly used is the angular frequency $\omega = 2\pi f$ so that

\begin{eqnarray}
  \Psi(x,t) = A\cos(k(x-vt)) = A\cos(kx-\omega t) \\
\end{eqnarray}

Sometimes and additional phase offset is used:
\begin{eqnarray}
  \Psi(x,t) = A\cos(k(x-vt)) = A\cos(kx-\omega t+\varepsilon) \\
\end{eqnarray}


\subsection{Some critical steps in dipped layer refraction seismics}
Goal: write $t^-$ as a function of velocities and dip. Isolate slope and simplfy y-offset.
$t^- = \frac{d^-}{v_1\cos(i_c)} + \frac{d^+}{v_1\cos(i_c)} + \frac{x\cos(\theta)-(d^-+d^+)\tan(i_c)}{v_2}$

\begin{eqnarray*}
  t^- = \frac{d^-}{v_1\cos(i_c)} + \frac{d^+}{v_1\cos(i_c)} + \frac{x\cos(\theta)-(d^-+d^+)\tan(i_c)}{v_2} \\
  = \frac{x\cos(\theta)}{v_2} + \frac{d^-}{v_1\cos(i_c)}  + \frac{d^+}{v_1\cos(i_c)} - \frac{(d^-+d^+)\tan(i_c)}{v_2} \\
  = \frac{x\cos(\theta)}{v_2} + \frac{d^-}{v_1\cos(i_c)}  + \frac{d^+}{v_1\cos(i_c)} - \frac{(d^-+d^+)\sin(i_c)}{\cos(i_c)v_2} \\
  = \frac{x\cos(\theta)}{v_2} + \frac{d^-}{v_1\cos(i_c)}  + \frac{d^+}{v_1\cos(i_c)} - \frac{(d^-+d^+)v_1}{\cos(i_c)v_2^2} \\
  = \frac{x\cos(\theta)}{v_2} + \frac{d^-}{v_1\cos(i_c)}  + \frac{d^+}{v_1\cos(i_c)} - \frac{(d^-+d^+)v_1^2}{v_1\cos(i_c)v_2^2} \\
  = \frac{x\cos(\theta)}{v_2} + \frac{1}{v_1\cos(\i_c)} (d^-+d^+-(d^-+d^+)\frac{v_1^2}{v_2^2}) \\
  = \frac{x\cos(\theta)}{v_2} + \frac{1}{v_1\cos(\i_c)} ((d^-+d^+)(1-\frac{v_1^2}{v_2^2})) \\
  = \frac{x\cos(\theta)}{v_2} + \frac{(d^-+d^+)\cos(\i_c)}{v_1}\\
\end{eqnarray*}
eliminate $d^+$ using use: $d^+ = d^- + x\sin(\theta)$ which is clear from geometry when you put $\theta$ at the surface.
\begin{eqnarray*}
  t^- = \frac{x\cos(\theta)}{v_2} + \frac{(d^-+d^+)\cos(i_c)}{v_1} \\
  = \frac{x\cos(\theta)}{v_2} + \frac{(2d^-+x\sin(\theta))\cos(i_c)}{v_1}\\
  = \frac{x\cos(\theta)}{v_2} + \frac{2d^-\cos(\i_c)}{v_1} +\frac{x\sin(\theta)\cos(i_c)}{v_1}\\
  = \frac{x\cos(\theta)\sin(i_c)}{v_2\sin(i_c)} + \frac{2d^-\cos(\i_c)}{v_1} +\frac{x\sin(\theta)\cos(i_c)}{v_1}\\
  = \frac{x\cos(\theta)\sin(i_c)}{v_1} + \frac{2d^-\cos(\i_c)}{v_1} +\frac{x\sin(\theta)\cos(i_c)}{v_1}\\
  = \frac{x\sin(\theta+i_c)}{v_1} + \frac{2d^-\cos(i_c)}{v_1}\\
\end{eqnarray*}
$\theta$ and $i_c$ are yet unknown but can be derived from the arcsin relation straightforward.
$$
\frac{1}{v_2^-}+\frac{1}{v_2^+} = \frac{1}{v_1}\sin(i_c)\cos(\theta) = \frac{2}{v_2}\cos(\theta)
$$
\end{document}




