\documentclass[a4paper,12pt]{article}
%All Layout Packages are defined in the Header.tex
\include{Header}
\newif\ifanswers
\answerstrue % comment out to hide answers
%Start the document here.
\begin{document}



\section{Exercises for Gravimetry}
\subsection{Determining the mean density of the Earth with your own gravimeter}
\noindent The vertical gravitational acceleration of a freely falling test object can be determined by measuring the traveltime for a known distance. Assuming that the radius of the Earth and the gravitational constant are known ($G\approx6.674\cdot10^{-11}$ m$^{3}$kg$^{-1}$s$^{-2}$,$R_E \approx 6370 $) the mass of the Earth can be determined with the basic principles derived in class.\\ 
\\
\noindent (a) Your task is to design a home-built free-fall gravimeter that does the job. You will quickly realize that the time measurement (i.e. the time from the start to the end of the fall) is one critical aspect in the system design. A helper tool that we suggest is the \textit{acoustic stopwatch} that can be accessed via a smartphone and the \textit{phyphox} app. Figure out an efficient way how the traveltime of a freely falling object can be determined with this type of acoustic trigger.\\
\\

\noindent (b) Collect a dataset of traveltimes and derive the mass of the earth. Provide an error estimate. Can you identify a measurement bias? How does your result compare to literature values? \\
\\
\noindent (c) Given your result, what is the mean density of the Earth? How does that compare to rock densities found at the Earth's surface? What is a main conclusion that you can draw from this?\\
 \\
\noindent (d) Post your result, your dataset and a picture of your system setup in the ILIAS forum. \\
 \\
\noindent (e) Extra: Are you a tinkerer/Bastler? If so, feel free to improve the system design, e.g., by using light barriers in combination with a raspberry pi nano. We are more than happy to buy material for you, the only constraint is that it works reliably and that it remains cheapish (let's say $<$100 Euro). If you succeed (success will be determined at the disgression of the instructor panel), you will win a gift certificate of 50 Euros that can be used in a Tübingen pub of your choice to celebrate your victory with your peers. Moreover, your system will be used for eternity for the following classes giving you much honor within the geo- and environmental student community (until somebody comes up with something better).


\subsection{The shell theorem}
\label{Sec:Shell{}}
\pagebreak
\subsection{Gravitational acceleration inside the Earth}
Knowing the gravitational acceleration \textit{inside} the Earth is important, e.g., for understanding processes related to mantel convection. \\
\\
\noindent (a) Use the consequences of the shell theorem to predict the gravitational acceleration $g(r)$ inside a spherical Earth. First assume that the Earth's density is constant, then that it decreases linearly with distance from the center. Assume that the density in the core is approximately $\rho_{core} = 13 \frac{g}{cm^3}$ and in the crust approximately $\rho_{crust} = 2.7 \frac{g}{cm^3}$.\\
\\
\noindent (b) Draw the results on a x-y graph on paper or using a software of your choice.\\
\\
\noindent (c) The PREM model provides an observationally constrained estimate of the density distribution inside the Earth. Knowing the shape of this profile, what is your guess of the gravitational acceleration inside the Earth? Why is it more difficult for you to calculate this quantitatively compared to the previous cases of constant and linearly varying density?\\
\\
\noindent (d) Extra: Calculate the gravitational acceleration inside the Earth based on the PREM model using Matlab/Python/Excel. The data can be found on Ilias.
\begin{center}
    \includegraphics[width=0.5\textwidth]{Figures/Gravimetry/Gravimetry01_PREM.png}
 \end{center}
\ifanswers
    \noindent \textbf{Solutions:}

    \noindent (a) At location $l<R$ inside the Earth, we do not need to worry about the mass distribution for distances $>l>R$ as these will cancel out due to the spherical symetry. Also the the acceleration will only have a radial component. Hence,
    \begin{equation}
        |\vec{g}(l)| = G\frac{M}{l^2}
    \end{equation}
    where $M$ is mass contained in the sphere with radius $l$. How does $M$ change as we change $l$? In this spherical symetry it is best to adopt spherical coordinates so that we can write: 
    \begin{eqnarray}
    M &=& \int \rho dV \\ 
    &=& \int_0^{\frac{\pi}{2}} d\theta \sin(\theta) \int_0^{2\pi}  d\phi \int_0^l dl\,\rho\,l^2 dV
    \end{eqnarray}
    this is a fairly complicated way of writing something that you know by heart (i.e. the density times the volume of a sphere), but this is a good oppertunity to make use of spherical coordinates. Most importantly, it is worthwhile to remember that the volume element $dV = l^2 \sin(\theta)drd\theta d\phi$ is quite different from the cartesian coordinates that you are used to otherwise. Also take a moment and understand why the integration limits are the way they are. Solving the integral step by step for the case of a constant density:
    \begin{eqnarray}
        M &=& \rho \int_0^{\frac{\pi}{2}} d\theta \sin(\theta)\int_0^{2\pi}  d\phi \int_0^l l^2 dl \\
         &=&\rho 2\pi \int_0^{\frac{\pi}{2}} d\theta \sin(\theta)\int_0^l l^2dl\\
         &=&\rho 4\pi \int_0^l l^2dl \\ 
         &=&\rho \frac{4}{3}\pi l^3
    \end{eqnarray}
    Plugging (7) into (1) then results for $l<R$:
    \begin{equation}
        |\vec{g}(l)| = G  \rho \frac{4}{3}\pi \frac{l^3}{l^2} = G  \rho \frac{4}{3}\pi l
    \end{equation}
    so the gravitational acceleration increases linearly with distance from the Earth's center as long as $l<R$ and $\rho$ is constant. A linear decrease of density with increasing distance can be parameterized as:
    \begin{equation}
        \rho(l) = \frac{\rho_{crust}-\rho_{core}}{R} (l-R) + \rho_{crust} = \frac{\rho_{crust} - \rho_{core}}{R}l+\rho_{core}
    \end{equation}
    As the density now changes with $l$ we cannot take it out of the integration. In particular for eq. (6) this means:
    \begin{eqnarray}
        M &=& 4\pi \int_0^l dl \left(\frac{\rho_{crust} - \rho_{core}}{R}l^3+\rho_{core}l^2\right)\\
          &=& 4\pi \left(\frac{\rho_{crust} - \rho_{core}}{4R}l^4+\frac{1}{3}\rho_{core}l^3 \right)
    \end{eqnarray}
    so that:
    \begin{eqnarray}
        |\vec{g}(l)| &=& 4 G \pi \frac{1}{l^2} \left(\frac{\rho_{crust} - \rho_{core}}{4R}l^4+\frac{1}{3}\rho_{core}l^3 \right) \\
        &=& 4 G \pi \left(\frac{\rho_{crust} - \rho_{core}}{4R}l^2+\frac{1}{3}\rho_{core}l \right)
    \end{eqnarray}
    \begin{center}
        \includegraphics[width=0.99\textwidth]{Figures/Gravimetry/Gravimetry01_GravityInsideEarth.png}
     \end{center}
     so that the gravitational acceleration now changes quadratically. Is this realistic? We will have to compare it with a realistic dataset such as the PREM model.

     \noindent Because we don't have an analytical expression for the PREM density model, we cannot pursue an analytical integration as done in the previous two cases. Hence we do it numerically:
     \lstinputlisting[language=matlab]{../Src/Exercises/Gravimetry/Gravimetry03_GravityInsideSphere.m}
\fi
\pagebreak


\subsection{Detection of a spherical object in the sub-surface}
\label{Sec:SphereInSubsurface}
\textbf{(a)} Consider a spherical object with radius $R$ located at depth $z$ below the surface, and a gravimeter that is moved along the surface across the anomaly (Fig.\ref{Fig:SphereInSubsurface}) Derive an analytical expression for the expected \textit{vertical} anomaly $g_z$ as a function of the lateral distance $x$ and depth $z$. Calculate the maximum anomalies that you would expect for some realistic settings (e.g., a limestone cave.)\\
\begin{figure}
\centering
\begin{minipage}[c]{0.5\textwidth}
\begin{center}
    \includegraphics[width=\textwidth]{Figures/Gravimetry/Gravimetry01_SphereSketch.png}
\end{center}
\caption{Sketch for problem \ref{Sec:SphereInSubsurface}. It is easiest to choose a coordinate system with the origin inside the subsurface object.}
\label{Fig:SphereInSubsurface}
\end{minipage}
\end{figure}

\ifanswers
    \noindent\fcolorbox{myblue}{lightgray}{\parbox{\textwidth}{
    \begin{center}
       \includegraphics[width=0.5\textwidth]{Figures/Gravimetry/Gravimetry01_SphereSketchSolutions.png}
    \end{center}
    \textbf{Solution:} Because we have a spherical object, we can use the shell theorem and treat it as a point mass. The gravitational force exhibited by the sphere is:
    $$
    \vec{g} = -G\frac{M}{r^2}\hat{r}
    $$
    where $\hat{r}$ is the unit vector pointing from the center of mass to our gravimeter at the surface. The minus sign makes sure that the force is attractive (and not repulsive). The mass anomaly $M$ formulated in a relative sense of the sphere is given by:
    $$
    M = \frac{4}{3}\pi R^3 \Delta \rho
    $$
    At distance $x$ the distance to the sphere is $r=\sqrt{x^2+z^2}$. The gravimeter only measures the vertical component of $\vec{g}$:
    $$
    g_z = -|\vec{g}|\cos\phi = -|\vec{g}|\frac{z}{r}
    $$
  
    which brings us to:
    \begin{align*}
    g_z &= -\frac{4}{3}\pi R^3 \Delta \rho \frac{G}{r^2} \frac{z}{r} \\
    &= -\frac{4}{3}\pi R^3 \Delta \rho \frac{G}{r^3} z \\
    & = -\frac{4}{3}\pi R^3 \Delta \rho G \frac{z}{(x^2+z^2)^\frac{3}{2}}
    \end{align*}
    We will encounter the maximum value of the anomaly at $x = 0$ (directly above the target):
    $$
    g_{z,max} = \frac{4}{3}\pi R^3 \Delta \rho G \frac{1}{z^2}
    $$
    which for the limestone cave example is approximately -0.35 mGal (see below).
    }}
    \pagebreak
\fi

\noindent \textbf{(b)} Use a piece of software of your choice (e.g., Excel, SciDAVis, Matlab, Python) and plot the expected vertical gravity anomaly for a specific setting as a function of lateral distance $x$. Visualize how this profile changes as you vary, e.g., the depth of the object. Label your axis and post a picture in the forum alongside with a comment which software you used.

\ifanswers  
    \lstinputlisting[language=matlab]{../Src/Exercises/Gravimetry/Gravimetry01_SphereVisualization.m}
    \begin{center}
        \includegraphics[width=0.5\textwidth]{Figures/Gravimetry/Gravimetry01_Visualization.png}
    \end{center}
\fi

\noindent\textbf{(c)} We now investigate how we can estimate the depth of the object from an anomaly as derived in (a,b). For this you have to derive an expression that links the half-width of the anomaly (i.e. $g_z = \frac{1}{2}g_{z,max}$ for $x=x_{1/2}$) to the depth $z$ of the object. Derive this relationship which has the form $z = c x_{1/2}$. Why is such a relationship be useful?


\noindent\fcolorbox{myblue}{lightgray}{\parbox{\textwidth}{
\textbf{Solution:} This requires some arithmetic manipulation to isolate $z$.
\begin{align*}
g_z &=\frac{1}{2}g_{z,max} \\
\rightarrow \frac{z}{(x^2_{1/2}+z^2)^\frac{3}{2}} &= \frac{1}{2z^2}\\
\rightarrow \frac{z^3}{(x_{\frac{1}{2}}+z^2)^\frac{3}{2}}&= \frac{1}{2}\\
\rightarrow \frac{1}{(\frac{x_{\frac{1}{2}}}{z^2}+1)^\frac{3}{2}}&= \frac{1}{2}\\
\rightarrow z = \frac{x_{1/2}}{\sqrt{2^{\frac{2}{3}}-1}}\approx 1.305 x_{1/2}
\end{align*}
Those expressions can be useful to estimate the depth of an object from an observed anomlay without using a full forward model. However, this only works if the object is indeed spherical (which we usually don't know necessarily ahead of time.) Similar estimates exists for other shapes (e.g., horizontal and vertical cylinders).
}}
\pagebreak
\subsection{Potential of an infinte plate (Bouger plate)}
\pagebreak


\subsection{Forward modelling and non-uniqueness in potential field methods}

\end{document}