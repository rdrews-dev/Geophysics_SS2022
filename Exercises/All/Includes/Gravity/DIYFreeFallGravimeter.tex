\subsection{Determining the mean density of the Earth with your own gravimeter}
\begin{tcolorbox}[enhanced jigsaw,breakable,pad at break*=1mm,
    colback=blue!5!white,colframe=burgundy,title=Group Work,
    watermark color=white]
    Self-organize in groups with 5-6 team members. Choose a group name and a team captain that communicates with the instructors. It is ideal if this group stays together throughout the term for the applied exercises. Make sure that your are inclusive during the group formation.
\end{tcolorbox}

  The vertical gravitational acceleration can be determined by measuring the traveltime of a freely falling object for a known distance. Assuming that the radius of the Earth and the gravitational constant are known ($G\approx6.674\cdot10^{-11}$ m$^{3}$kg$^{-1}$s$^{-2}$, $R_E \approx 6370 $) the mass of the Earth can be determined with the basic principles derived in class.

  \begin{enumerate}[label=(\alph*)]
    \item Your task is to design a home-built free-fall gravimeter that does the job. You will quickly realize that the time measurement (i.e. the time from the start to the end of the fall) is one critical aspect in the system design. A helper tool that we suggest is the \textit{acoustic stopwatch} that can be accessed via a smartphone and the \textit{phyphox} app. Figure out an efficient way how the traveltime of a freely falling object can be determined with this type of acoustic trigger. (Tip: You may need a hammer, a metal bar and a bar clamp for your acoustic trigger. Be creative.)
    
    \item Collect a dataset of traveltimes and derive the mass of the earth. Provide an error estimate. Can you identify a measurement bias? How does your result compare to literature values? 
    
    \item Given your result, what is the mean density of the Earth? How does that compare to rock densities found at the Earth's surface? What is a main conclusion that you can draw from this?

    \item Post your result, your dataset and a picture of your system setup in the ILIAS forum. 

    \item Extra: Are you a tinkerer/Bastler? If so, feel free to improve the system design, e.g., by using light barriers in combination with a raspberry pi nano. We are more than happy to buy material for you, the only constraint is that it works reliably and that it remains cheapish (let's say $<$100 Euro). \textbf{If you succeed, you will win a gift certificate of that can be used in a Tübingen pub of your choice to celebrate your victory with your peers.} Moreover, your system will then be used for eternity for the following classes giving you much honor within the geo- and environmental student community.
  \end{enumerate}
  \ifanswers
            \begin{tcolorbox}[enhanced jigsaw,breakable,pad at break*=1mm,
                colback=blue!5!white,colframe=babyblueeyes,title=Solutions,
                watermark color=white]
                I made a video with a possible setup here: 
                \begin{center}
                    \href{https://www.youtube.com/watch?v=q5SdryMp8iw&feature=youtu.be}{Link to Video} 
                \end{center}
                \url{https://www.youtube.com/watch?v=q5SdryMp8iw&feature=youtu.be}
            \end{tcolorbox}
  \fi