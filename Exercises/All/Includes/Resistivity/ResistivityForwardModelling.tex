\subsection{Forward Modelling of a vertical electrical sounding survey}
(a) Use the attached file \textit{RD\_VES\_ForwardModel\_Ex6.ipynb} which contains a Jupyter Notebook with PyGimli as discussed in the latest video. Set-up a sub-surface model with three different resistivities for cases:
\begin{itemize}
    \item $ \rho_1 > \rho_2 > \rho_3$
    \item $ \rho_1 < \rho_2 < \rho_3$
    \item $ \rho_1 < \rho_2, \rho_2 > \rho_3$
\end{itemize}
Also explore possibilities of four layers cases. This type of forward modelling can be useful for your applied exercises in geoelectrics.

(b) Change the Jupyter notebook code so that you can visualize output from two forward runs (use a copy and past where you can). Illustrate to different sub-surface models which result in a similar vertical electrical sounding observations. Memorize that does ambiguities (same as, e.g., for gravity surveys) are very prevalent in geophysical applications.
\ifanswers
    \begin{tcolorbox}[enhanced jigsaw,breakable,pad at break*=1mm,
    colback=blue!5!white,colframe=babyblueeyes,title=Solutions,
    watermark color=white]
    %\lstinputlisting[language=python]{../../Src/Resistivity/RD_VES_ForwardModel_Ex6Solution.ipynb}
    ../../Src/Resistivity/RD\_VES\_ForwardModel\_Ex6Solution.ipynb
    \end{tcolorbox}
\fi