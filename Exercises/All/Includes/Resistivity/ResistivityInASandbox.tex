\subsection{Resistivity Survey in a Sandbox}
\begin{figure}[th]
    \centering
    \includegraphics*[width=0.5\textwidth]{Figures/Resistivity/Sandbox.png}
    \caption{Toy setup for understanding the principles of the resistivity method.}
\end{figure}
\begin{tcolorbox}[enhanced jigsaw,breakable,pad at break*=1mm,
    colback=blue!5!white,colframe=burgundy,title=Group Work,
    watermark color=white]
    Self-organize with your groups to do this during our contact time on Thursdays. Ideally two to three groups should do the experiment in a total of 2h. We will assign two sessions for that. After you are done, disconnect all cables and keep everything as you found it for the next group. \textbf{When using the battery with open-end cables, be careful that you do not short them.}
\end{tcolorbox}
(\textit{practical}) Use a battery, four electrodes and two multimeters to determine the apparant resistivity of sand in a sandbox. You can align the electrodes in a co-linear Wenner array and also check how/if the resistivity changes with variable array types and moisture content. Post your results in the forum and compare them with  the literature values. Don't overestimate the accuracy here, it is more about internalizing the measurement principle and to strengthen your cabling skills.  All material (Fig. 1) will be provided on-site. 

\ifanswers
    \begin{tcolorbox}[enhanced jigsaw,breakable,pad at break*=1mm,
    colback=blue!5!white,colframe=babyblueeyes,title=Solutions,
    watermark color=white]
    Discussions on site and video online.
    \end{tcolorbox}
\fi
