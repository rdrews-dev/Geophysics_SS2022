\subsection{Reflection seismics}
You get a desperate Email from an all friend of yours:\\

\textit{My dearest friend,}\\

\textit{I just got hired in a geophysical prospecting company and earn good money. Unfortunately, I never took a geophysics lecture and I can't handle math at all (which sort of makes me wonder why they hired me in the first place.) My supervisor handed me some data (cf. Figs. 1\& 2) of a test seismic survey. They only did one explosion and wanted to estimate in this shot gather what the subsurface looks like, before collecting more data.  I am supposed to give a presentation next week answering the following questions:}
\textit{
\begin{itemize}
    \item Which seismic wave types are visible in the shot gather?
    \item What are the seismic velocities and over which depth intervals do they change?
    \item Does the survey confirm our expectations (from the geologic context) that we have two stratigraphic units at this location which are horizontal?
    \item What are the interval velocities of the individual layers? (some new content)
    \item Can we explain all signatures in the shot gather? (some new content)
\end{itemize}
}
\textit{My supervisor suggested that I start with the analysis of linear features and then move on to the shallowest and deepest reflection hyperbolas. Apparantly the root-mean-square velocity $v_{RMS}$ and the respective interval velocities $v_i$ at layers $i$ traversed by time $\Delta t_i$ are connected like that:}
$$
v_{RMS} = \sqrt{\frac{\sum_i v_i^2 \Delta t_i}{\sum_i \Delta t_i}}
$$

\textit{To be honest, I don't even know what she is talking about. Could you please help me out and send me some drawings + calculations that I can use in the presentation? This will not be forgotton. I wish I had taken more rigorous lectures during my studies.} \\

\textit{Regards, Your Friend}


\pagebreak
\begin{figure}
\includegraphics*[width=\linewidth]{Figures/Seismics/ShotGather_smallgain.pdf}
\caption{Shot gather for shot at postion 0, geophonespacing is 20 m.}
\end{figure}

\pagebreak
\begin{figure}
\includegraphics*[width=\linewidth]{Figures/Seismics/ShotGather_largergain.pdf}
\caption{Shot gather for shot at postion 0, geophonespacing is 20 m. Compared to Fig.1 the amplitudes are scaled so that weaker signals are more apparant.}
\end{figure}
test.
\pagebreak 
\ifanswers
    \begin{tcolorbox}[enhanced jigsaw,breakable,pad at break*=1mm,
    colback=blue!5!white,colframe=babyblueeyes,title=Solutions,
    watermark color=white]
    
        \begin{center}
        \includegraphics*[width=0.8\linewidth]{Figures/Seismics/ShotGather_solutions.pdf}
        \end{center}
    Linear features show direct wave travelling with $v_1 = 700 $ ms$^{-1}$ (magenta lines). Knickpoint at around 400 m profile distance. Then arrivals of headwave with $v_2 = 1000 $ ms$^{-1}$. Hyperbolic reflection hyperbola (green lines) with $t_0 = 0.228$ s. Analysis of velocities using $t^2-x^2$ method or normal moveout is straightforward (see Matlab code). Lowest hyperbola can be done in the same way. Middle hyperbola (red lines) is a multiple from the first reflection interface with a traveltime: 
    $$
      t_{multiple} = \frac{4}{v_1}\sqrt{d_1^2+\frac{x}{4}^2}  
    $$
    which can be derived from the geometry knowing that the reflection point of the multiple is at x/4 and not x/2 as is the case for the primary reflection. The interval velocities of the lower layer can be derived by inverting the Dix-Dürrbaum equation provided in the letter.
    \begin{center}
        \includegraphics*[width=0.5\linewidth]{Figures/Seismics/TsqXsq.pdf}
        \end{center}
        \lstinputlisting[language=matlab]{../../Src/Seismics/RD_Functions/RD_ShotRecordHorizontalLayering.m}
\end{tcolorbox}
\fi

