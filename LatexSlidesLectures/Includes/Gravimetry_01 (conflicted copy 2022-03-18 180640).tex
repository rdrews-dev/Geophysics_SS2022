\begin{frame}
    \begin{PointThree}{Learning Goals}
      \alert{Learning goals today:}
      \begin{itemize}
        \item The gravitational force, its potential field, and how to measure it.
        \item The gravitational field of the Earth.
        \item .... 
      \end{itemize}
    \end{PointThree}
    \end{frame}
    
\begin{frame}
    \begin{PointSix}{Example: Global variability}
        \includegraphics[width=0.99\textwidth]{Figures/Gravity/Exported/Grace_JPLCaltect_FODT10_WithoutPeople.png}
    \end{PointSix}
\end{frame}
    
\begin{frame}
    \begin{PointSix}{Example: Global variability}
    \centering
    \small Your mass is constant but your weight is not.
    \includegraphics[width=0.99\textwidth]{Figures/Gravity/Exported/Grace_JPLCaltect_FODT10_WithPeople.png}
    \end{PointSix}
\end{frame}
    

\begin{frame}
\begin{ThreeCols}{What is a force?}{
    \includegraphics[width=0.8\textwidth]{Figures/Gravity/Exported/Newton_PD_GJohnson.png} \centering \tiny [Newton (1642-1726) / G. Johnson.]}
    \scalebox{1.5}{%
        $
        \vec{F} = m \vec{g}
        $
    }
    \scalebox{0.6}{\parbox{\linewidth}{
        \begin{align*}
        &\vec{F}:\,\text{Force}\,(\text{N};\,\text{kg}\,\text{m}\,\text{s}^{-2})\\
        &\vec{a}:\,\text{Acceleration}\,(\text{m}\,\text{s}^{-2})\\
        &\text{m}:\,\text{Mass (kg)}
        \end{align*}
    }}
\end{ThreeCols}
\end{frame}

\begin{frame}
    \begin{ThreeCols}{The gravitational force}{
        \includegraphics[width=0.8\textwidth]{Figures/Gravity/Exported/Newton_PD_GJohnson.png} \centering \tiny [Newton (1642-1726) / G. Johnson.]}
        \scalebox{1.5}{%
            $
            \vec{F} = G\frac{mM}{r^2}\hat{r}
            $
        }
        \scalebox{0.6}{\parbox{\linewidth}{
            \begin{align*}
            &G=6.674 \cdot 10^{-11}\,\text{(}\,m^3 kg^{-1} s^{-2}\text{)}\\
            &\hat{r}:\,\text{unit vector}\\
            &r:\,\text{distance between point masses}
            \end{align*}
        }}
        \begin{tikzpicture}
            \coordinate (A) at (1,2);
            \coordinate (B) at (2,4);
            
            \draw [fill=white] (A) circle (8pt) node [left,xshift=-0.5cm] {M};
            \draw [fill=white] (B) circle (4pt) node [left,xshift=-0.5cm] {m};
            
               
            \draw[-latex,thick,Karminrot,->] (A) -- (B) node[midway,left,rotate=0] {$r$};
            \draw [-latex,thick, Karminrot] (B) -- (A);
            \end{tikzpicture}
    \end{ThreeCols}
    \end{frame}

    \begin{frame}
        \begin{PointSix}{Example: The gravitational constant}
            \includegraphics[width=0.99\textwidth]{Figures/Gravity/Exported/Cavendish_PNAS1798.png}
            \centering
            \tiny Cavnedish, PNAS, 1798
        \end{PointSix}
    \end{frame}
    \begin{frame}
        \begin{PointSix}{Example: The gravitational constant}
            \includegraphics[width=0.99\textwidth]{Figures/Gravity/Exported/MeasuringG_Westphal_Nature2021.png}
            \centering
            \tiny Westphal et al., Nature, 2021\\
            \normalsize G is the worst known constant in physics. Why?
        \end{PointSix}
    \end{frame}   

\begin{frame}
  \begin{PointSix}{Example: Measuring acceleration}
    \begin{minipage}[t]{0.3\textwidth}
      \includegraphics[width=\textwidth]{Figures/General/ScarySmiley_Kindpng.png}
      \end{minipage}\begin{minipage}[]{0.7\textwidth}
    \centering
    \begin{align*}
    &\vec{F} = m \vec{g} \\
    &\vec{F} = G\frac{mM}{r^2}\hat{r} \\
    &\rightarrow \vec{g} = G\frac{M}{r^2}\hat{r} \\ 
    &\rightarrow \frac{d^2\vec{x}}{dt^2} = G\frac{M}{r^2}\hat{r} \\
    \end{align*}
  \end{minipage}
    \centering
    This is a differential equation.
    
  \end{PointSix}
\end{frame}

\begin{frame}
  \begin{PointSix}{Example: Measuring acceleration}
    \begin{minipage}[t]{0.3\textwidth}
      \includegraphics[width=\textwidth]{Figures/General/IcandothisSmiley_Kindpng.png}
      \end{minipage}\begin{minipage}[]{0.7\textwidth}
    \centering
    \begin{align*}
    &\frac{d^2\vec{x}}{dt^2} = G\frac{M}{R_E^2} \approx const. \\
    \end{align*}
  \end{minipage}
   % \centering
    At the Earth's surface ($R_E$) g is close to constant and only vertical. (Later we will see that none of this is not quite true).
    
  \end{PointSix}
\end{frame}

\begin{frame}
    \begin{PointSix}{Example: Measuring acceleration}
    \begin{tikzpicture}
        \begin{axis}[
          rdstyle,
          xlabel=time (s),
          ylabel={$g (m s^{-2})$},
          xmin=0,
          xmax=10,
          xtick={0,2,...,10},
          color=white,
          width=9cm,
        ] 
          \addplot[domain=0:10,samples=10,color=Karminrot,line width=0.5mm] {x*0 +9.81}; 
          \node[] at (axis cs: 5,11) {$g = \frac{d^2}{dt^2}x(t)=\frac{GM}{R_e^2}\approx const.$};
        \end{axis}
      \end{tikzpicture}
    \end{PointSix}   
\end{frame}

\begin{frame}
    \begin{PointSix}{Example: Measuring acceleration}
    \begin{tikzpicture}
        \begin{axis}[
          rdstyle,
          xlabel=time (s),
          ylabel={$v (m s^{-1})$},
          xmin=0,
          xmax=10,
          xtick={0,2,...,10},
          ytick=9.81,
          yticklabels={$c_1$},
          color=white,
          width=9cm,
        ] 
          \addplot[domain=0:10,samples=10,color=Karminrot,line width=0.5mm] {x + 9.81}; 
          \node[] at (axis cs: 5,20) {$v = \int g dt=\frac{d}{dt}x(t)=\frac{GM}{R_e^2}t+c_1$};
        \end{axis}
      \end{tikzpicture}
    \end{PointSix}   
\end{frame}

\begin{frame}
    \begin{PointSix}{Example: Measuring acceleration}
    \begin{tikzpicture}
        \begin{axis}[
          rdstyle,
          xlabel=time (s),
          ylabel={$x (m)$},
          xmin=0,
          xmax=10,
          xtick={0,2,...,10},
          ytick=9.81,
          yticklabels={$c_2$},
          color=white,
          width=9cm,
        ] 
          \addplot[domain=0:10,samples=10,color=Karminrot,line width=0.5mm] {x^2 + 10}; 
          \node[] at (axis cs: 5,100) {$x(t) = \int v(t) dt=\frac{GM}{2R_e^2}t^2+c_1t+c_2$};
        \end{axis}
      \end{tikzpicture}
    \end{PointSix}   
\end{frame}
\begin{frame}

  \begin{PointSix}{Example: Measuring acceleration}
       \begin{align*}
         & x(t) = \frac{GM}{2R_e^2}t^2+c_1t+c_2 \\
       \end{align*}
       \begin{itemize}
        \item Setting, e.g., $c1=0$ (initial velocity) and $c_2=0$ (initial position) is quite convenient.   
        \item This is the principal of a free-fall gravimeter.
      \end{itemize}
  \end{PointSix}
  
\end{frame}
\begin{frame}
  \begin{PointSix}{Exercises: Group-Work Thursdays}
    \begin{itemize}
      \item Thanks to the Greeks we know the radius $R_E$ for the Earth. However, its mass was unknown for a while.
      \item Go ahead and determine the mass of the Earth M with your Smartphone!
      \item \alert{There is an important first-order finding in Earth Sciences that you can (re-) discover. Which one?}
    \end{itemize}
      
  \end{PointSix}
\end{frame}

\begin{frame}
  \begin{PointSix}{Beyond point masses}
    \begin{tikzpicture}
      \coordinate (A) at (8,0);
      \coordinate (B) at (4,-3.5);
      \coordinate (C) at (5,-3.5);
      \coordinate (D) at (6,-3.5);
      \draw [thick] (0,0.0) -- (8,0.0);
      % drawing the node with shape=rectangle and anchor=center
      \node [draw, Karminrot, thick, shape=rectangle, minimum width=0.25cm, minimum height=0.25cm, anchor=center] at (B) {};
      \draw [->, thick] (0,0.0) -- (B) node[xshift=0.3cm,yshift=0.3cm,midway,left,rotate=-30] {$\vec{r}$};;
  
      \node[yshift=0.3cm] at (A) {\small Surface};

    \end{tikzpicture}
    $$
      \vec{F} = G\frac{dM}{r^2}\hat{r}
    $$
  \small For a small mass dM the point mass approximation holds.
  \end{PointSix}

\end{frame}



\begin{frame}
  \begin{PointSix}{Beyond point masses}
    \begin{tikzpicture}
      \coordinate (A) at (8,0);
      \coordinate (B) at (4,-3.5);
      \coordinate (C) at (5,-3.5);
      \coordinate (D) at (6,-3.5);
      \draw [thick] (0,0.0) -- (8,0.0);
      % drawing the node with shape=rectangle and anchor=center
      \node [draw, Karminrot, thick, shape=rectangle, minimum width=0.25cm, minimum height=0.25cm, anchor=center] at (B) {};
      \foreach \i in {0,2,...,8}
      {
        \draw [->, thick] (0+\i,0.0) -- (B);
      }
      \node[yshift=0.3cm] at (A) {\small Surface};

    \end{tikzpicture}
    $$
      \vec{F} = G\frac{dM}{r^2}\hat{r}
    $$
    \small Profiling across a sub-surface target results in a gravity anomaly ($\rightarrow$ Exercises).
  \end{PointSix}
\end{frame}

\begin{frame}
  \begin{PointSix}{Beyond point masses}
    \begin{tikzpicture}
      \coordinate (A) at (8,0);
      \coordinate (B) at (4,-3.5);
      \coordinate (C) at (5,-3.5);
      \coordinate (D) at (6,-3.5);
      \draw [thick] (0,0.0) -- (8,0.0);
      \node[yshift=0.3cm] at (A) {\small Surface};

      \foreach \i in {-2,-1.5,...,2}
      {
        \foreach \j in {0,0.5}
        {
          \node [draw, Karminrot, thick, shape=rectangle, minimum width=0.25cm, minimum height=0.25cm, anchor=center] at (4-\i,-3.5-\j) {};
          \draw [->, thick] (0,0.0) -- (4-\i,-3.5-\j) ;

        }
      }
      
    \end{tikzpicture}
    $$
    \vec{F}(\vec{r}) = \sum_i G\frac{dM_i}{r_i^2}\hat{r_i}
    $$
    \small For $i$ point masses the effect adds up.
  \end{PointSix}
\end{frame}

\begin{frame}
  \begin{PointSix}{Beyond point masses}
    \begin{tikzpicture}
      \coordinate (A) at (8,0);
      \coordinate (B) at (4,-3.5);
      \coordinate (C) at (5,-3.5);
      \coordinate (D) at (6,-3.5);
      \draw [thick] (0,0.0) -- (8,0.0);
      \node[yshift=0.3cm] at (A) {\small Surface};

      \foreach \i in {-2,-1.5,...,2}
      {
        \foreach \j in {0,0.5}
        {
          \node [draw, Karminrot, thick, shape=rectangle, minimum width=0.25cm, minimum height=0.25cm, anchor=center] at (4-\i,-3.5-\j) {};
          \draw [->, thick] (2,0.0) -- (4-\i,-3.5-\j) ;

        }
      }  
    \end{tikzpicture}
    $$
      \vec{F}(\vec{r}) = \sum G\frac{dM_i}{r_i^2}\hat{r_i}
    $$
  \end{PointSix}
  
\end{frame}

\begin{frame}
  \begin{PointSix}{Beyond point masses}
    \begin{tikzpicture}
      \coordinate (A) at (8,0);
      \coordinate (B) at (4,-3.5);
      \coordinate (C) at (5,-3.5);
      \coordinate (D) at (6,-3.5);
      \draw [thick] (0,0.0) -- (8,0.0);
      \node[yshift=0.3cm] at (A) {\small Surface};
      \foreach \i in {-2,-1.5,...,2}
      {
        \foreach \j in {0,0.5}
        {
          \node [draw, Karminrot, thick, shape=rectangle, minimum width=0.25cm, minimum height=0.25cm, anchor=center] at (4-\i,-3.5-\j) {};
          \draw [->, thick] (4,0.0) -- (4-\i,-3.5-\j) ;

        }
      }
    \end{tikzpicture}
    $$
    \vec{F}(\vec{r}) = \sum G\frac{dM_i}{r_i^2}\hat{r_i}
    $$
  \end{PointSix}
\end{frame}


\begin{frame}
  \begin{PointSix}{Beyond point masses}
    \begin{tikzpicture}
      \coordinate (A) at (8,0);
      \coordinate (B) at (4,-3.5);
      \coordinate (C) at (5,-3.5);
      \coordinate (D) at (6,-3.5);
      \draw [thick] (0,0.0) -- (8,0.0);
      \node[yshift=0.3cm] at (A) {\small Surface};

      \foreach \i in {-2,-1.5,...,2}
      {
        \foreach \j in {0,0.5}
        {
          \node [draw, Karminrot, thick, shape=rectangle, minimum width=0.25cm, minimum height=0.25cm, anchor=center] at (4-\i,-3.5-\j) {};
          \draw [->, thick] (7,0.0) -- (4-\i,-3.5-\j) ;

        }
      }
    \end{tikzpicture}
    $$
    \vec{F}(\vec{r}) = \sum G\frac{dM_i}{r_i^2}\hat{r_i}
    $$
  \end{PointSix}
\end{frame}


\begin{frame}
  \begin{PointSix}{Beyond point masses}
    \begin{tikzpicture}
      \coordinate (A) at (8,0);
      \coordinate (B) at (4,-3.5);
      \coordinate (C) at (5,-3.5);
      \coordinate (D) at (6,-3.5);
      \draw [thick] (0,0.0) -- (8,0.0);
      \node[yshift=0.3cm] at (A) {\small Surface};
      \node[yshift=-1.8cm,xshift=-4cm] at (A) {$\vec{F}(\vec{r}) = G \int \rho \frac{1}{r^2}\hat{r}dV$};
      
      % 
      \foreach \i in {-2,-1.5,...,2}
      {
        \foreach \j in {0,0.5}
        {
          \node [draw, Karminrot, thick, shape=rectangle, minimum width=0.25cm, minimum height=0.25cm, anchor=center] at (4-\i,-3.5-\j) {};
          %\draw [->, thick] (7,0.0) -- (4-\i,-3.5-\j) ;
        }
      }
      \node [draw, Karminrot, thick, shape=rectangle, minimum width=4.5cm, minimum height=1.25cm, anchor=center] at (4,-3.75) {}; 
    \end{tikzpicture}

    \only<1>{\small The summation can be replaced by an integration over a volume enclosing a continuous density.}\only<2>{\small The integration is a triple integral. Integration limits and coordinates depend on the viewpoint. Example is a Bouger plate, in general not easy to solve ($\rightarrow$ Exercises).} 
      %\only<1>{The summation can be replaced by an integration over a volume enclosing a continuous density.} d $\rho$}\only<2>{Test,}d
  \end{PointSix}
\end{frame}

\begin{frame}
  \begin{PointSix}{Example: Shell}
    \includegraphics[width=0.99\textwidth]{Figures/Gravity/Exported/Shell-diag_reversed_CCBYSA4_0_Xaonon.png}
    \tiny [Xaononl CC BY-SA 4.0]

    \small Newton's shell theorem solves the volume integral inside and outside spherical objects ($\rightarrow$ Ex.-Discussion)
  \end{PointSix}

\end{frame}

\begin{frame}
  \begin{PointSix}{Newton's Shell Theorem}
   \begin{itemize}
      \item The field outside a shell is the same as the one from an equivalent point mass
      \item The field inside a shell is zero. Everywhere.
   \end{itemize}
\end{PointSix}
\end{frame}

\begin{frame}
\begin{PointSix}{Numerical forward modelling ($\rightarrow$ Ex)}
  \includegraphics[width=0.8\textwidth]{Figures/Gravity/Exported/ForwardModelPrismReversed.png}
\end{PointSix}
\end{frame}

\begin{frame}
  \begin{tikzpicture}
    \begin{axis}[
        xmin = -4, xmax = 4,
        ymin = -4, ymax = 4,
        zmin = 0, zmax = 1,
        axis equal image,
        xtick distance = 1,
        ytick distance = 1,
        view = {0}{90},
        scale = 1.25,
       % title = {\bf Vector Field $F = [-y,x]$},
        height=7cm,
        xlabel = {$x$},
        ylabel = {$y$},
        colormap/viridis,
        %colorbar,
        %colorbar style = {
        %    ylabel = {Vector Length}
        %}
    ]
        \addplot3[
            point meta = {sqrt(x^2+y^2)},
            quiver = {
                u = {-x/sqrt(x^2+y^2)^2},
                v = {-y/sqrt(x^2+y^2)^2},
                scale arrows = 0.7,
            },
            quiver/colored = {mapped color},
            -stealth,
            samples = 10,
            domain = -4:4,
            domain y = -4:4,
        ] {0};
        \node [draw, Karminrot, thick, shape=rectangle, minimum width=4.5cm, minimum height=1.25cm, anchor=center] at (4,-3.75) {}; 

        % \addplot3[contour gnuplot={number=50,labels=false, draw color=blue},thick,]
        % {
        %   abs(((1)/sqrt((x-2)^2+y^2) - (1)/sqrt((x+2)^2+y^2)))<4 ? ((1)/sqrt((x-2)^2+y^2) - (1)/sqrt((x+2)^2+y^2)) : NaN 
        % };
    \end{axis}
  \end{tikzpicture} 
  \end{frame}

\begin{frame}
  \begin{tikzpicture}
    \begin{axis}[
      title={$x \exp(-x^2-y^2)$},
      domain=-2:2,enlarge x limits,
      view={0}{90},
      ]
      \addplot3[contour gnuplot={number=14},thick]
      {exp(-x^2-y^2)*x};
    \end{axis}
  \end{tikzpicture}
\end{frame}


\end{document}​




